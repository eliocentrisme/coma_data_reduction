% -*- coding: UTF-8; -*-
%ne pas modifier ou enlever la ligne ci-haut
\documentclass[11pt,letterpaper]{article}

%Document préparé par David Lafrenière, pour le cours PHY1234.

%Pour langue et caractères spéciaux
%\usepackage[french]{babel} 
\usepackage[T1]{fontenc}
\usepackage{lmodern}
\usepackage[utf8]{inputenc}
\usepackage{siunitx}
\usepackage{array}
\usepackage{multirow}
\usepackage{subcaption}
\usepackage{float}
\usepackage{amsmath}
\usepackage{amssymb}
\usepackage{hyperref}
\usepackage{braket}
%\usepackage[french]{babel} 
%Pour ajuster les marges
\usepackage[top=2cm, bottom=2cm, left=2cm, right=2cm]{geometry}

%pour inclure des graphiques
\usepackage{graphicx}
\usepackage[export]{adjustbox}

%Pour inclure des adresse web
\usepackage{url}

\usepackage{soul, color}

%Pour deux colonnes
%\usepackage{multicol}

%pour le résumé
\usepackage{abstract}
	\renewcommand{\abstractnamefont}{\normalfont\bfseries}
	\renewcommand{\abstracttextfont}{\normalfont\itshape}

%pour inclure les codes en annexe
\usepackage{fancyvrb}
\usepackage{listingsutf8}
\usepackage{color}
\lstset{inputencoding=utf8/latin1,numbers=left,numberstyle=\footnotesize,frame=single,commentstyle=\it\color{blue},keywordstyle=\bf\color{red}}

\usepackage{abstract}
	\renewcommand{\abstractnamefont}{\normalfont\bfseries}
	\renewcommand{\abstracttextfont}{\normalfont\itshape}


\usepackage[
backend=biber,
style=authoryear,
sorting=nyt
]{biblatex}
\addbibresource{coma.bib}

%Page titre
\title{\textbf{Literature review of galaxy clusters and the Coma cluster}}
\author{
    Elio Desbiens, \textit{Physics Department, Université de Montréal}\\
}
\date{\today}

\begin{document}

%\twocolumn[
\maketitle


\begin{onecolabstract}
%Ce projet a pour but de ... \\

%\par nous donner une bonne note :)

\noindent This literature review provides a concise overview of galaxy clusters, including their nature, general structure, and significance in physics research. It briefly discusses radio observations and outlines current knowledge of the Coma Cluster (Abell 1656), highlighting the reasons for its particular scientific interest and recent research developments. The review concludes by emphasizing the need for continued observations and further investigation in the field of galaxy clusters.


\vspace{4mm} %
\end{onecolabstract}
%]

\section{Galaxy Clusters: An Overview} \label{sec:galaxy_clusters}
\subsection{Defining Galaxy Clusters} \label{subsec:galaxy_clustersr_what}
\par Galaxy clusters are large-scale structures composed of up to a few thousand galaxies, with typical total masses of $\sim 10^{15}M_{\odot}$. The \textit{intra-cluster medium} (ICM)—a very hot ($T\sim 10^8 K$), low-density ($n \sim 10^{-3} cm^{-3}$), magnetized plasma—comprises about 15\% of the total mass, while dark matter accounts for roughly 80\%. Notably, most of the baryonic mass in a galaxy cluster resides in the ICM. The ICM emits thermal Bremsstrahlung radiation at X-ray wavelengths, making it observable through X-ray telescopes.

\medskip
\par Galaxy clusters are typically found at the nodes of cosmic filaments composed primarily of the warm-hot intergalactic medium (WHIM), which contains approximately half of the universe’s baryons. Clusters form through the accretion of WHIM and through mergers of galaxy groups and clusters. These structures are held together by gravity, which is strong enough to overcome cosmic expansion. Clusters can be either \textit{relaxed}—undisturbed by recent mergers—or \textit{merging}, where ongoing mergers inject energy into the system. This energy is dissipated through shocks and turbulence, heating the ICM. It may also be transferred into relativistic particles such as cosmic rays (CRs), observable via diffuse synchrotron radio emission, and magnetic fields, which are more challenging to detect (\cite{brunetti_cosmic_2014}).

\medskip
\par The appearance of galaxy clusters varies significantly with the observational wavelength (figure \ref{fig:optical-radio}), each offering unique insights into their physics. A broad multi-wavelength approach is essential to confirm findings and deepen our understanding of the underlying physical processes. However, this review focuses on the radio emissions from galaxy clusters. \cite{van_weeren_diffuse_2019} proposed three main classes of diffuse radio sources in clusters. Classification must be approached with care, as projection effects can obscure the true nature of these sources. The first class includes \textit{radio halos}, which are subdivided into giant halos, mini-halos, and intermediate or hybrid types. These are thought to arise from second-order Fermi re-acceleration of seed electrons (\cite{pasini_ultra-low_2024}), a process involving the stochastic scattering of particles by magnetic inhomogeneities generated by merger-driven turbulence in the ICM (\cite{van_weeren_diffuse_2019}). This model is supported by low-frequency ($<$ 150 MHz) observations revealing steep energy spectra with $\alpha < -1.5$ for most halos studied by \cite{pasini_ultra-low_2024}, where $\alpha$ is the radio spectral index discussed in section \ref{sec:radio}.
\begin{figure}[t]
	\centering
	\includegraphics[width=\textwidth]{opt_x_radio.png}
	\caption{Optical, X-ray and radio views of the galaxy cluster Abell 2744. Figure from \cite{van_weeren_diffuse_2019}.}
	\label{fig:optical-radio}
\end{figure}

\medskip
\par Radio halos typically follow the baryonic mass distribution of the ICM and exhibit smooth, regular morphologies unassociated with specific shocks. This spatial non-locality supports the re-acceleration model but could also be consistent with emission from secondary electrons (e.g., pion decay products). A correlation has been observed between cluster temperature and spectral index, with hotter clusters tending to exhibit flatter average spectral indices (\cite{giovannini_radio_2009}). Giant radio halos are linked to cluster mergers and provide critical observational evidence for the origin of diffuse radio emission and the theoretical connections between cosmic ray acceleration and ICM physics (\cite{brunetti_cosmic_2014}). Mini-halos, in contrast, are found in cool-core regions of relaxed clusters.

\medskip
\par The second category comprises cluster radio shocks, or \textit{radio relics}, which have arc-like morphologies and are typically located in the cluster outskirts. In figure \ref{fig:coma_radio}, the diffuse emission in the lower-right region corresponds to the Coma cluster radio shock, as discussed in section \ref{sec:coma_cluster_what}. These relics are generated by shock waves from cluster mergers. They can form in opposing directions in the equatorial plane perpendicular to the merger axis, or along the axis itself, sometimes resulting in double radio shocks (\cite{van_weeren_diffuse_2019}). Due to their formation mechanisms, relics often display transverse brightness asymmetry and high degrees of polarization. Their steep spectra make them difficult to detect at higher frequencies (see section \ref{sec:radio}). As with halos, the visibility of radio relics is highly sensitive to projection effects and the orientation of the merger.

\medskip
\par The third category includes revived AGN \textit{fossil plasma sources}, such as phoenices and GReETs (gently re-energized tails). These sources are challenging to distinguish from radio relics because confirming their nature requires detecting shocks in mm or X-ray wavelengths, followed by corresponding radio observations (\cite{van_weeren_diffuse_2019}). AGN fossil plasma is a candidate source for the seed electrons necessary for forming radio halos and relics. These sources are generally smaller, weaker in radio power, more morphologically diverse, and located closer to the cluster center than relics. Unlike radio shocks, phoenices do not exhibit a clear connection to shock waves (\cite{van_weeren_diffuse_2019}).
% ajouter qqch sur greet

\medskip
\par Cosmic ray protons (CRp) in galaxy clusters can interact hadronically with ICM protons, producing pions that decay into gamma-rays. These gamma-rays, in principle, offer insights into the role of CRp and secondary electrons in radio emission and cluster dynamics. However, as of 2024, no definitive gamma-ray detections have been made (\cite{osinga_probing_2024}), except for the Coma cluster, where limited statistics prevent firm conclusions about the emission’s origin (\cite{baghmanyan_detailed_2022}). Previous studies cited in \cite{brunetti_cosmic_2014} suggest that CRp energy must be less than $\sim 1\%$ of the ICM’s total thermal energy in the cluster core—contrary to earlier expectations—though CRp may contribute more significantly at the cluster periphery.

\begin{figure}[t]
	\centering
	\includegraphics[width=\textwidth]{synchrotron_spectra.png}
	\caption{Synchrotron spectra for different populations of electrons. Left: standard electron population; center: aged electron population without injection of new particles; right: aged electron population with injection of new particles. Figures from \cite{pizzo_tomography_2010}.}
	\label{fig:synchrotron_spectra}
\end{figure}

% DSA shock models

\subsection{Importance of Galaxy Clusters} \label{subsec:galaxy_clusters_why}
\par Galaxy clusters are the building blocks of the universe; they form its large-scale structure. Being able to observe them at different redshifts is being able to know the history of the formation and the evolution of galaxy clusters. Their study provides a unique opportunity to investigate particle acceleration under extreme conditions inaccessible to terrestrial experiments. In these environments, relativistic particle acceleration occurs in collisionless, large-scale shocks characterized by high temperature-to-magnetic-field ratios and low Mach numbers (\cite{van_weeren_diffuse_2019}). Thus, galaxy clusters are essential laboratories for understanding the origin of cosmic rays, including potentially ultra-high-energy ones (\cite{condorelli_impact_2023}).

\medskip
\par Additionally, studying galaxy clusters enhances our understanding of their interconnections, as filaments—key to cluster formation—link them. These insights contribute to a deeper comprehension of the large-scale structure and evolution of the universe, which in turn provides essential clues about the primordial universe (\cite{malavasi_cosmic_2023}).

\begin{figure}[t]
    \centering
    \begin{subfigure}[c]{0.40\textwidth}
        \centering
        \includegraphics[width=\textwidth]{coma_optical.pdf}
		\caption{False color image of the HERON Coma field, in $g$ and $r$ bands, with some bright galaxies identified. Figure from \cite{jimenez-teja_deep_2024}}
		\label{fig:coma_optical}
    \end{subfigure}
	\hspace{3pt}
    \begin{subfigure}[c]{0.40\textwidth}
        \centering
        \includegraphics[width=0.9\textwidth]{coma_radio.jpg}
		\caption{Infrared (white) and radio (red-orange scale) image of the Coma cluster. Figure from \cite{bonafede_coma_2022-1}.}
		\label{fig:coma_radio}
    \end{subfigure}
       \caption{The Coma cluster. Up is North, left is East.}
       \label{fig:coma}
\end{figure}

\section{Probing Galaxy Clusters with Radio Observations}  \label{sec:radio}
% synchrotron radiation and radio spectra
The presence of diffuse non-thermal emission in galaxy clusters, detectable through radio observations, serves as a tracer of relativistic particles (cosmic rays) and large-scale magnetic fields. Charged particles moving through a magnetic field are deflected, and this deviation causes them to emit radiation in a cone of half-angle $\sim 1/\gamma$ (\cite{favart_1ere_2023}), perpendicular to the direction of acceleration. This emission is known as \textit{synchrotron radiation}. In galaxy clusters, the velocities of the electrons and the strength of the magnetic fields are such that synchrotron radiation is emitted at radio wavelengths (\cite{pizzo_tomography_2010}), making radio astronomy an interesting way of probing the physics of galaxy clusters. In an isotropic and homogeneous environment, the total intensity spectrum of electron synchrotron radiation follows a power-law distribution of the form
\begin{equation}
	S(\nu) = \nu^{-\alpha}
\end{equation}
where $\alpha$ is the spectral index mentioned earlier. Moreover, the observed synchrotron spectrum is the cumulative result of the spectra of all contributing particles. Since an electron’s energy decreases over time, the spectrum steepens beyond a certain critical frequency as the electrons age and lose energy. In scenarios where new particles are continuously injected, the steepening is less pronounced, resulting in a lower spectral index. Any of these three scenarios—also illustrated schematically in figure \ref{fig:synchrotron_spectra}—may be observed in diffuse radio emission sources. Sources with steeper radio spectra are more difficult to observe at higher frequencies. On one hand, they are fainter because their total flux is lower. On the other hand, sources with flatter spectra dominate in brightness and can overshadow fainter emissions, introducing bias in both data acquisition and analysis. Synchrotron radiation also allows for the estimation of magnetic field strength via rotation measures, though such measurements are difficult to obtain.
\medskip
\par Another important effect to take into account is the SZ effect predicted in 1970 by Sunyaev \& Zeldovich, where thermal electrons in the ICM interact with the CMB through inverse Compton scattering. This effect is redshift independent making it a powerful observational tool for the ICM (\cite{van_weeren_diffuse_2019}).
\medskip
\par Radio observations are necessary to the study of galaxy clusters because of the close link they have with relativistic particles and magnetic fields. They are our way of studying non-thermal emissions in the ICM, completing the picture made by X-ray, optical and SZ observations. The correlation between different bandwidths observations can help us constrain model parameters of particle (re-)acceleration in galaxy clusters as well as understand better the formation of galaxy clusters.
% inverse compton

\section{The Coma Cluster} \label{sec:coma_cluster}
\subsection{Current Understanding of the Coma Cluster} \label{sec:coma_cluster_what}
The Coma cluster is a galaxy cluster originally named Abell 1656, with redshift $z = 0.023$, situated at R.A. $12^{\text{h}}59^{\text{m}}48.73^{\text{s}}$ and Dec. +27\textdegree $58'50''$ (\cite{bilton_kinematics_2018}). The Coma cluster contains about two thousand galaxies, divided among $\sim 40$ distinct groups (\cite{jimenez-teja_deep_2024}). It is widely accepted that the Coma cluster is currently in a turbulent phase in which NGC 4889 and NGC 4874, the brightest cluster galaxies (BCGs), are in the process of merging. The presence of two BCGs—an uncommon feature among clusters—supports this idea of a recent or ongoing merger (\cite{jimenez-teja_deep_2024}, \cite{simionescu_thermodynamics_2013}). Additional support comes from a subcluster located to the southwest of the main cluster, which contains NGC 4839 and is slowly falling into the cluster potential (\cite{jimenez-teja_deep_2024}, \cite{simionescu_thermodynamics_2013}). This subcluster is connected to a filament itself connecting the Coma cluster with the cluster Abell 1367 as shown by X-ray observations in \cite{mirakhor_complete_2020}. Figure \ref{fig:coma_optical} shows the Coma cluster in the \textit{g} and \textit{r} bands, while figure \ref{fig:coma_radio} presents it in radio and infrared wavelengths.

\medskip
\par From low-frequency radio observations (\cite{bonafede_coma_2022-1}), several diffuse radio emission sources have been identified, as shown in figure \ref{fig:coma_radiolabels}. At the center, a radio halo of approximately 2 Mpc is observed. It appears larger than in previous high-frequency observations. The inner region of this halo, referred to as the "halo core", is brighter and contains several luminous filaments. The outskirts of the halo core exhibit lower surface brightness, though they appear brighter toward the west. Also to the west, a sharp boundary known as the "halo front" coincides with a shock front identified in X-ray and SZ observations (\cite{simionescu_thermodynamics_2013}, \cite{planck_collaboration_planck_2013}). In \cite{bonafede_coma_2022-1}, it was found that the radio halo is likely larger than previously reported. Its spectrum also appears flatter than in past studies at similar wavelengths, although there is a moderate steepening toward the cluster center.

\begin{figure}[t]
	\centering
	\includegraphics[width=0.7\textwidth]{coma_radiolabels.jpg}
	\caption{LOFAR image of the Coma cluster with labelled diffuse radio emission sources. Figure from \cite{bonafede_coma_2022-1}.}
	\label{fig:coma_radiolabels}
\end{figure}

\medskip
\par To the southwest of the radio halo lies a 1.1 Mpc radio relic that appears more filamentary and patchy. It is connected to the northeast with NGC 4839 via a structure known as a "bridge", and to the southwest with the narrow-angle tail (NAT) galaxy NGC 4789. The radio relic supports the hypothesis of a recent or ongoing merger. To the south of the galaxy cluster, some radio emission seems to originate from NGC 4849, extending roughly $\sim 650$ kpc in the north-south direction. To the east of the radio halo, another region of radio emission is labeled "candidate remnant". Depending on the resolution, it may be interpreted either as an accretion relic (\cite{pizzo_tomography_2010}) or as emission from a radiogalaxy, although no core has been detected (\cite{bonafede_coma_2022-1}). Finally, to the northeast and just outside the cluster's virial radius, there is diffuse emission that could be confirmed as an accretion relic with additional data. The virial radius is defined as the radius at which the mean total density is 100 times the critical density of the universe at the redshift of the cluster, according to \cite{bonafede_coma_2022-1}.

\medskip
\par With high-sensitivity and high-resolution observations, it is possible to identify extended radio sources ($\gtrsim 12.6$ kpc), and ram-pressure-stripped galaxies in the Coma cluster, the majority of which are steep-spectrum sources (\cite{lal_high-resolution_2022}). These sources, like all radio sources, undergo synchrotron and inverse Compton cooling, although spectral steepening may also result from adiabatic cooling and the decline of magnetic fields around the cores of the sources.

% \begin{figure}[t]
% 	\centering
% 	\includegraphics[width=0.7\textwidth]{coma_flux_density.png}
% 	\caption{}
% 	\label{fig:coma_flux_density}
% \end{figure}

% not directly associated with an active galaxy (JR1979)
% \subsection{Previous Coma studies}
% \begin{table}[h!]
%     \centering
%     \begin{tabular}{c|c|c|c|c|c}
%         \hline 
% 		\hline
%         Frequency (MHz) & Telescope & RMS & Flux & Dynamic range & Reference\\
%         \hline
%         1400 & SRT && & & \cite{murgia_sardinia_2024} \\
%         6600 & SRT & 0.33 mJy/beam  & & & \cite{murgia_sardinia_2024} \\
%         250-500 & uGMRT  & 21.1 $\mu$Jy/beam & & 5300 & \cite{lal_upgraded_2020} \\
% 		1050-1450 & uGMRT & 12.7 $\mu$Jy/beam & & 1700 & \cite{lal_upgraded_2020} \\
% 		1410 & GBT & 6 mJy/beam & & & \cite{brown_diffuse_2018}\\
% 		352 & WSRT &  && & \cite{brown_diffuse_2018}\\
% 		144 & LOFAR & 15 mJy/beam??? & && \cite{bonafede_coma_2022-1} \\
% 		30.9 & CLRO &  & 49 $\pm$ 10 Jy &  & \cite{henning_309_1989} \\
% 		43.0 &  CLRO &&&& \cite{henning_309_1989} \\
% 		73.8 &  CLRO &&&& \cite{henning_309_1989} \\
% 		151 & CLFST & 12 mJy/beam & 7.2 $\pm$ 0.8 Jy && \cite{cordey_radio_1985} \\ % peak @ 3.20 Jy
% 		326 & WSRT & 1.2 mJy/beam && 3000:1 & \cite{venturi_high-sensitivity_1990} \\
% 		1400 & VLA &  0.1 mJy/beam &&& \cite{venturi_high-sensitivity_1990} \\
% 		408 & DRAO & 8 mJy/beam & 2.0 $\pm$ 0.2 Jy && \cite{kim_halo_1990} \\
% 		1420 & DRAO &&&& \cite{kim_halo_1990} \\
% 		1380 & VLA & 30 $\mu$Jy/beam & 0.53 $\pm$ 0.05 Jy & 3000:1 & \cite{kim_halo_1990} \\
% 		1630 & VLA &&&& \cite{kim_halo_1990} \\
% 		430 & NAIC & 25 mJy/beam &&& \cite{hanisch_diffuse_1980} \\
% 		1400 & NAIC & 10 mJy/beam &&& \cite{hanisch_diffuse_1980} \\
% 		608.5 & WSRT & 0.5 mJy/beam & 1.2 $\pm$ 0.3 Jy & & \cite{giovannini_halo_1993} \\
% 		1400 & Effelsberg & 7 mJy/beam &  && \cite{deiss_large-scale_1997} \\
% 		2675 & Effelsberg &  &  && \cite{thierbach_diffuse_2003} \\
% 		4850 & Effelsberg &&&& \cite{thierbach_diffuse_2003} \\
% 		2700 & Effelsberg & 1.5 mJy/beam & 0.08 $\pm$ 0.02 Jy && \cite{schlickeiser_diffuse_1987} \\ %peak 28mJy/beam
%         \hline
% 		\hline
% 	\end{tabular}
% 	\caption{Effelsberg 100m telescope}
% \end{table}


\subsection{Why the Coma Cluster Matters}  \label{sec:coma_cluster_why}
The Coma cluster is nearby and X-ray bright, making it easily observable in that bandwidth. It is also dynamically active and has a non-cool core, making it a valuable subject for studying cluster physics, as it exhibits signs of a recent merger. Currently, it is the only known candidate for high-energy gamma-ray emission (\cite{baghmanyan_detailed_2022}). Detecting gamma rays is essential for understanding the role of cosmic rays and pion decay products in cluster dynamics. This would help validate models currently inferred only from radio observations. Fortunately, the Coma cluster is one of the most thoroughly studied galaxy clusters—along with the Virgo and Perseus clusters (\cite{lal_high-resolution_2022})—thanks to its relative proximity ($z = 0.023$) and the fact that it has been observed across nearly all wavelengths, facilitating structural studies. It also has the best-studied magnetic field.

% \section{Recent Advances in Coma Cluster Research} \label{sec:recent_work}
% % about Coma part. in radio
% In \cite{bonafede_coma_2022-1}, it was found that the radio halo is probably larger than previously reported by earlier studies. Its spectrum is also flatter than previously at similar wavelengths, but there is a moderate steepening towards the cluster center. 



\section{Open Questions and Future Work} \label{sec:why_more}
In \cite{bonafede_coma_2022-1}, a scenario describing the recent past and present state of the Coma cluster is proposed, although the data is not yet sufficient to confirm it. With new data and ongoing studies, it will be possible to further constrain theoretical models, eventually converging toward one that best fits the observations—thereby improving our understanding of galaxy cluster physics. For instance, \cite{xrism_collaboration_xrism_2025} found a velocity power spectrum that is steeper than predicted by the classic model of steady-state turbulence, which had previously been assumed for the Coma cluster. Meanwhile, \cite{malavasi_cosmic_2023} aims to model the accretion of matter through filaments connecting the cluster to the cosmic web by comparing simulations with observations. As noted by \cite{mirakhor_complete_2020}, the contribution of the northern and northeastern filaments connecting the Coma cluster to the cosmic web remains unclear. One current theory is that energy flowing from the southwestern filament is preventing the cluster from reaching a relaxed state.

\medskip
\par Other promising avenues include examining how the halo size varies with frequency and conducting more detailed studies of the halo front and the accretion relic. Although the halo front has been confirmed to coincide with a shock front observed in X-ray and SZ data, no definitive conclusions can yet be drawn. If confirmed, the accretion relic would represent the first detection of particle acceleration from an accretion shock, in line with the diffusive shock acceleration (DSA) model. A more detailed investigation of the candidate remnant identified by \cite{bonafede_coma_2022-1} is also essential to improving our understanding of the Coma cluster.

\newpage

\printbibliography

\end{document}
